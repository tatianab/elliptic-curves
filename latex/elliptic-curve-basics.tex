%\chapter{Elliptic Curves}%
\label{sec:elliptic-curves}

\section{Introduction to Elliptic Curves}
% Weierstrass equation
\begin{defn}
Let $K$ be a field. An $\textbf{elliptic curve $E$ over $K$}$ is defined by an equation:
$$E : y^2 + a_1xy + a_3y = x^3 + a_2x^2 + a_4x + a_6$$
where $a_1, a_2, a_3, a_4, a_6 \in K$ and the $\textbf{discriminant}$ $\Delta$\footnote{If you must know, $$\Delta = -d^2_2d_8 - 8d^3_4 - 27d_6^2 + 9d_2d_4d_6$$ } is non-zero. This equation is called a $\textbf{Weierstrass equation}$.

When char$(K) \neq 2, 3$, we can change variables
to arrive at the simplified Weierstrass equation:
$$ E : y^2 = x^3 + ax + b $$
where $a, b \in K$. 
\end{defn}

% L-rational points
\begin{defn}
With $K$ and $E$ defined as above, the set of $\textbf{L-rational points}$ on $E$ for any extension $L$ of $K$ is the set of pairs $(x, y) \in L \times L$ that
satisfy $E$, together with $\OO$, the point at infinity.

The set of L-rational points is denoted $E(L)$.
\end{defn}

An elliptic curve can be defined over any field $K$, but in cryptography we generally restrict $K$ to be a finite field $F_q$ where
$q = p^n$, for $p$ prime and $n \in \Z_{>0}$. In this paper we will restrict ourselves even further to prime fields
$\F_p$ where $p \neq 2, 3$, and to an infinite field $\Q_p$ (the $p$-adics), which have characteristic 0. This means we will always be able to use the
simplified equation given in the first definition above.

% Trace of Frobenius
\begin{defn}
Let $E$ be an elliptic curve over a finite field $\finfield$. The $\textbf{trace of Frobenius}$ t is defined by:
$$ \#E(\finfield) = q + 1 - t, $$
where $\#E(\finfield)$ is the number of elements in $E(\finfield)$.
\end{defn}

\begin{rmk}
The trace of Frobenius is equal to one if and only if $E(\finfield)$ has exactly $q$ elements. This has important implications
for cryptography, as we will see.
\end{rmk}

\subsection{The group law}
\todo[inline, caption={Geometric group law}]{Describe geometric group law and how it motivates algebraic one. Possibly add
figures to motivate this.}
\begin{defn} Let $E(K)$ be an elliptic curve over a field $K$ with $char(K) \neq 2, 3$ defined by $y^3 =x^3  ax + b$ with point at infinity $\OO$. Let
 $P = (x_P, y_P)$ and $Q = (x_Q, y_Q)$ be points on $E(K)$. Then:
\begin{enumerate}
\item(Identity.)
$\OO + P = P$ and $P + \OO = P$.
\item(Additive inverses.)
The additive inverse of $P$, denoted $-P$, is in $E(K)$, and
$P + (-P) = \OO$.
\item{(Point Doubling.)}
If P is not its own inverse, then $P + P = 2P = (x_{2P}, y_{2P})$ where
$$x_{2P} = N^2 - 2 x_P, $$
$$y_{2P} = N(x_P - x_{2P}) - y_P,$$
$$N = \frac{3x_P^2 + a}{2y_P}.$$
\item(Point Addition.)
If  $P \neq Q$ and $P \neq -Q$ then
$P + Q = R = (x_{R}, y_{R})$ where
$$x_{R} = M^2 - x_P - x_Q,$$
$$y_{R} = M * (x_P - x_{R}) - y_P,$$
 $$M = \frac{y_Q - y_P}{x_Q - x_P}.$$

\end{enumerate}

\end{defn}$E(K)$ is an abelian group under this group law.
\begin{note}
Let $P$ be a point on an elliptic curve. We use the notation $[n]P$ to denote scalar multiplication of $P$ by a non-zero integer $n$. In other words, $$[n]P = P + P + ... + P \  \mathrm{(n \ times)}.$$
\end{note}

\subsection{Projective space and projective coordinates}


\subsection{The Elliptic Curve Discrete Logarithm Problem}
The interesting thing about elliptic curves with regards to cryptography
is that their structure can be used to construct a "one-way" function.\footnote{In this case one-way is in quotes because the problem is only hard in
certain cases, and there is no guarantee that there is not some clever way to render the
general problem easy.}
 A one-way function is one that is easy to perform but hard to undo (i.e., it is difficult to retrieve the input,
given an output). 
\begin{defn}[Elliptic Curve Discrete Logarithm Problem]
Let $E$ be an elliptic curve over a finite field $\F_q$, and let $P$ be a point on $E$. Suppose we have a point $Q$
on $E$ that is some scalar multiple of $P$, i.e.,
$$[n]P = Q, \ \ \ \  n \in \NN. $$
The \textbf{elliptic curve discrete logarithm problem} (ECDLP) is to determine the natural number $n$, given $E$, $P$
and $Q$.
\end{defn}

On the other hand, the problem of determining $[n]P$ given $n$ and $P$ is not hard. One simple (to describe)
way to do this is by using successive squaring.

\begin{defn}[Successive Squaring for Elliptic Curves].
Suppose we have an elliptic curve $E(K)$, a point $P \in E(K)$ and a nonnegative integer $n$.
We can compute $[n]P$ recursively by calling SuccessiveSquare($n$, $P$). The analogue of
squaring for elliptic curves is doubling.\\ \\
SuccessiveSquare(nonnegative integer $m$, point on elliptic curve $Q$): \todo{Fix formatting here}
\begin{enumerate}
\item
If m = 0, return $\OO$.
\item
If m = 1, return $Q$.
\item
If $n$ is even, return $[m/2](2Q)$ by calling SuccessiveSquare$(n / 2, 2Q)$.
\item
If $n$ is odd, return $[(m - 1)/2](2Q)$ by calling SuccessiveSquare( $(m-1) / 2, 2Q$) and add $Q$.
\end{enumerate}
This asymptotic run-time of this algorithm is $O(\log{n})$, a considerable improvement on
brute force computation of $P + P + P + ...$ which takes $O(n)$ time. There are even faster
algorithms for this computation that we do not describe here.
\end{defn}


